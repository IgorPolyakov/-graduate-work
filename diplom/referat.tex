\newpage
\ESKDthisStyle{plain}
\paragraph{\hfill Реферат \hfill}
Выпускная квалификационная работа \ESKDtotal{page} с., \ESKDtotal{figure} рис., \ESKDtotal{table} табл., \ESKDtotal{bibitem} источников, \ESKDtotal{appendix} прил., 8 л. графического материала.

Целью настоящей работы является разработка программного обеспечения (ПО) для оценки деформаций поверхностей твердых тел, а также проведение исследований алгоритмов и методов как на искуственных, так и на реальных оптических изображениях.

В работе исследовано влияние метода интерполяции итеративного подхода с субпиксельной точностью на расчет оптического потока(векторного поля), а также исследовано влияние предварительной обработки изображений на построение полей векторов смещений.

Исследованны методы для поиска смещений больше чем 20px. 

 

Проект выполнен с использованием следующих средств разработки: языка программирования C++(Qt), среды разработки QtCreator 3, Sublime 3. Система контроля версий git.

Программный продукт, разработанный в ходе данной работы, представлен в приложении.

Отчет выполнен согласно ОС ТУСУР 01-2013 при помощи текстового процессора \LaTeX.