\subsection {Оценка быстродействия}
Для проведения расчётов использовали ПЭВМ со следующими характеристиками.

Аппаратная составляющая:
\begin{itemize}
\item процессор Intel$^{\textregistered}$~Core\texttrademark~i3 370M @ 2,4 ГГц, 64-бит;
\item оперативная память 4 Гб, 1067 МГц;
\item материнская плата Aspire 5733z;
\item жёсткий диск SSD Smartbuy 120 Гб;
\item файловая система ext4, 107 Гб.
\end{itemize}

Программная составляющая:
\begin{itemize}
\item операционная система Linux Mint 3.19-18;
\item версия cmake 3.2.1;
\item Qt version 5.2.1;
\item версия компилятора gcc (Ubuntu 4.8.2-19ubuntu1) 4.8.2;
\item оболочка для среды рабочего стола Cinnamon 2.4.8.
\end{itemize}

Для исследования точности и быстродействия алгоритмов были использованы описанные ранее серии изображений. В  качестве параметра точности считалось среднеквадратическое отклонение:
$$\sigma=\sqrt{\frac{1}{n}\sum_{i=1}^n\left(x_i-\bar{x}\right)^2}.$$