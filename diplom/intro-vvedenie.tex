\setcounter{figure}{0}
\newpage
\section{Введение}
\addcontentsline{toc}{section}{Введение}
%Данное программное обеспечение разрабатывается для задач оценки деформации твёрдого тела.

%Решение задач оценки деформации охватывают широкий спектр применения от биологической до аэрокосмической и в масштабах от микроскопии до крупных промышленных структур.

%В задачах автоматизации данное программное обеспечение может использоваться для автоматизации измерений при проведении экспериментальных исследований по оценке и прогнозированию поведения объектов и конструкций при воздействии нагрузок в полевых условиях и на испытательных стендах.

Использование систем технического зрения (СТЗ) для определения параметров деформации представляется в настоящее время как наиболее перспективное, поскольку позволяет предсказывать места локализации деформации и концентрации критических напряжений задолго до возникновения микротрещин.

СТЗ оптико-телевизионных измерительных комплексов по принципу действия относятся к обзорно-сравнительным системам к классу корреляционных зрительных систем (КЗС) \cite{korrel_robot}. СТЗ данного типа находят широкое применение в различных системах промышленной автоматизации, в том числе и в робототехнических системах.

%Корреляционные зрительные системы используют корреляционно-экстремальный метод обработки зрительной информации и по сути представляют собой разновидность корреляционно-экстремальных систем (КЭС). Известно, что работа КЭС базируется на распознавании объекта и определении его искомых характеристик путем обработки информации, представленной в виде реализаций случайных функций. Термин КЭС объясняется тем, что по принципу действия подавляющее большинство известных КЭС представляют собой системы экстремального управления.

Экспериментальное изучение процессов пластической деформации, развивающихся в материалах и при различных условиях эксплуатации, позволит дать рекомендации по новым методам упрочнения, созданию оптимальных типов покрытий, соотношению механических свойств покрытия и матрицы (основы), составам композиционных материалов. Проведение таких экспериментальных исследований на мезоуровне требовало создания новых аппаратных и программных средств, в качестве которых могут выступать методы и средства технического зрения, включающие автоматизированные измерительные комплексы, способные оперативно выполнять обработку больших объемов видеоинформации, производить высокоточные измерения, качественно и наглядно представлять полученные результаты.

Целью настоящей работы является разработка программного обеспечения (ПО), рассмотрение алгоритмов для оценки деформаций поверхностей твердых тел, т.е. для получения качественной и количественной оценки процессов, развивающихся в деформируемом твердом теле на мезоуровне, а также проведение исследований алгоритмов и методов на реальных оптических изображениях.

Область применения ПО:
1. экспериментальное исследование механизмов деформации и разрушения структурно-неоднородных материалов с целью последующего компьютерного моделирования их структуры и свойств;
2. аттестация режимов формирования защитных и упрочняющих покрытий с целью корректировки режимов их нанесения;
3. тарировка приборов неразрушающего контроля и дефектоскопии.

%Внедрение работы. Созданное программное обеспечение включено в состав оптико-телевизионного комплекса "TOMSC" и используется для проведения исследований различных материалов и сплавов в ИФПМ СО РАН.