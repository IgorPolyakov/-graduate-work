Qt - это кросс-платформенная библиотека C++ классов для создания графических пользовательских интерфейсов (GUI) от фирмы Digia. Эта библиотека полностью объектно-ориентированная, что обеспечивает лёгкое расширение возможностей и создание новых компонентов. Ко всему прочему, она поддерживает огромнейшее количество платформ.

Qt позволяет запускать написанное с его помощью ПО в большинстве современных операционных систем путём простой компиляции программы для каждой ОС без изменения исходного кода. Включает в себя все основные классы, которые могут потребоваться при разработке прикладного программного обеспечения, начиная от элементов графического интерфейса и заканчивая классами для работы с сетью, базами данных и XML. Qt является полностью объектно-ориентированным, легко расширяемым и поддерживающим технику компонентного программирования.

Список использованных классов фреймворка Qt:
\begin{itemize}
\item QDebug;
\item QDir;
\item QFile;
\item QTextStream;
\item QSrting;
\item *QImage;
\item QTime.
\end{itemize}

Класс QDebug обеспечивает выходной поток для отладочной информации.

Класс QDir обеспечивает доступ к структуре каталогов и их содержимого.

Класс QFile предоставляет интерфейс для чтения и записи файлов.

Класс QTextStream предоставляет удобный интерфейс для чтения и записи текста.

Класс QString обеспечивает строку символов Unicode.

Класс QImage предоставляет аппаратно-независимую работу с изображениями, даёт прямой доступ к данным каждого пикселя, и может быть использован в качестве устройства рисования. На ранних стадиях создание ПО это класс использовался в качестве основного инструмента для работы с изображениями, но впоследствии при переходе на основные модули библиотеки DV стал использовался объект Data2D, и методы работы с ними.

Класс QTime обеспечивает функции таймера. В проекте используется для оценки быстродействия разрабатываемого модуля при разных входных параметрах~\cite{qt}. 

Список стандартных библиотек используемых в проекте:

\begin{itemize}
\item getopt.h;
\item iostream;
\item math.h;
\item stdlib.h;
\item vector.
\end{itemize}

getopt библиотечная функция, специально разработанная для того чтобы облегчить обработку входных команд. 

iostream заголовочный файл с классами, функциями и переменными для организации ввода-вывода в языке программирования C++. Он включён в стандартную библиотеку C++. Название образовано от Input/Output Stream (``поток ввода-вывода'').

math.h — заголовочный файл стандартной библиотеки языка программирования С, разработанный для выполнения простых математических операций.

stdlib.h — заголовочный файл стандартной библиотеки языка Си, который содержит в себе функции, занимающиеся выделением памяти, контроль процесса выполнения программы, преобразования типов и другие.

vector.h — это замена стандартному динамическому массиву, память для которого выделяется вручную, с помощью оператора new.

Список классов и методов библиотеки DV используемых в проекте:
\begin{itemize}
\item Data2D
\item Matx22d
\item Vec2d
\item VF2d
\end{itemize}