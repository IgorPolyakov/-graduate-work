\subsection{Обоснование необходимости проводимого исследования}

Настоящая дипломная работа исследует реализацию дифференциального алгоритма Лукаса-Канаде ?и его вариации? для поиска смещений по парам изображений. На основе полученных векторных полей смещений можно строить карты поверхностной деформации твердого тела. 

Целью данной работы является создание программного обеспечения для анализа оптических изображений поверхности материалов, выявление скорости и точности работы алгоритма Лукаса-Канаде в зависимости от окна поиска и метода аппроксимации при суб-пиксельном смещении изображения.

Продолжение исследований по данной тематике может привести к ХХХХХХХХХХХХХХХХХХХХХХХХХХХХХХХХХХХХХХХХХХХХХХХХХ.
\subsection{Планирование комплекса работ по разработке программного обеспечения}
Основными задачами планирования работ являются:
\begin{enumerate}
\item определение объема предстоящих;
\item распределение объема работ на взаимосвязанные последовательные этапы;
\item установление сроков выполнения работ;
\item определение необходимых, для выполнения планируемых работ денежных, материальных и трудовых ресурсов.
\end{enumerate}
При выполнении дипломной работы было задействовано два человека:
\begin{enumerate}
\item руководитель (рук.);
\item разработчик (разр.).
\end{enumerate}
Руководитель выполняет контроль выполнения различных этапов работ, согласованность этапов выполнения работ между собой, корректирует действия разработчика, дает рекомендации по выполнению тех или иных работ. Разработчик реализует тот объем работ, который установлен руководителем в соответствие с техническим заданием.

Месячный оклад студента в ТУСУР равен 1430 рублей, с учетом 21 рабочих дней в месяце, и 8 часового рабочего дня, стоимость одного часа работ равна 8,5 рублей. Месячный оклад руководителя д.н., профессора в университете равен 14800 рублей, с учетом 24 рабочих дней, и 6 часового рабочего дня, стоимость одного часа работ равна 102,8 рубля. [Приказ ректора от 22.03.2013 г. № 3106. С изменениями от 09.12.2013 г. № 14249]

График выполнения работ приведен в таблице 1.1.

Зная длительность цикла каждого этапа и возможность их параллельно-последовательного выполнения, можно рассчитать срок завершения планируемых работ и составить ленточный и сетевой графики плана их выполнения. Поскольку работа не требует большого состава исполнителей, то ограничимся ленточным графиком планирования, представленным в виде таблицы 1.2.

Таблица 1.2 – Ленточный график загрузки участников работ

\subsection{Определение сметной стоимости проекта}
\subsubsection{Общие положения}
Смета затрат для данной работы состоит из расходов, которые включают в себя следующие статьи:
\begin{enumerate}
\item затраты на оборудование и амортизацию;
\item расходы на оплату труда и отчисления на социальные нужды;
\item затраты на основные и вспомогательные материалы;
\item затраты на электроэнергию.
\end{enumerate}
\subsubsection{Затраты на оборудование и амортизацию}
Основным оборудованием при проведении работы являются компьютер и принтер, которые постановлением Правительства Российской Федерации от 1.01.02 г. N 1 отнесены ко второй амортизационной группе – «имущество со сроком полезного использования свыше 2 лет до 3 лет включительно». Месячная норма амортизации составляет 2,8\% и для компьютера, и для принтера.
Результаты расчетов амортизационных отчислений приведены в таблице 1.3.
Таблица 1.3 – Смета затрат на оборудование
\subsubsection{Расходы на оплату труда и отчисления на социальные нужды}
Статья затрат учитывает выплаты по заработной плате за выполненную работу, исчисленные на основании тарифных ставок и должностных окладов в соответствии с принятой в организации-разработчике системой оплаты труда. В этой статье также отражаются премии, надбавки и доплаты за условия труда, оплата ежегодных отпусков, выплата районного коэффициента и некоторые другие расходы. Отчисления на социальные нужды учитывают страховые взносы.
Результаты расчета расходов на оплату труда участников проекта представлены в таблице 1.4.
Таблица 1.4 – Расчет расходов на оплату труда участников проекта
\subsubsection{Затраты на основные и вспомогательные материалы}
Статья включает расходы по приобретению и доставке основных и вспомогательных материалов, необходимых для опытно-экспериментальной проработки решения, для изготовления макета или опытного оборудования. Сюда включаются и стоимость необходимых материалов для изготовления образцов и макетов, и материалов необходимых для оформления требуемой документации. 
Размер транспортно-заготовительных расходов (ТЗР), определяемый в процентах от стоимости, примем 10\%. Стоимость вспомогательных материалов принимается 10\% от стоимости основных материалов с учетом ТЗР. Результаты расчета стоимости материалов представлены в таблице 1.5.
Таблица 1.5 – Смета затрат на основные и вспомогательные материалы
\subsubsection{Расходы на электроэнергию}
Статья включает затраты по электроэнергии на технологические нужды. В настоящее время тариф на электроэнергию для населения г. Томска на 2015 год составляет 2,7 руб./ кВт ч. Введенный приказом от 26.12.2014 №6/9 (691) ``О тарифах на электрическую энергию для населения и потребителей, приравненных к категории население по Томской области на 2015 год'', принятый департаментом тарифного регулирования Томской области.
Результаты расчетов приведены в таблице \ref{tab:zatr_elek}.
\begin{longtable}[h!]{|*6{m{0.14\textwidth}|}}
\caption{Затраты на электроэнергию}
\label{tab:zatr_elek}
\hline
Наименование оборудования & Количество, шт. & Потребляемая мощность, кВт & Часы работы & Тариф за,1 час, руб. & Стоимость электроэнергии, руб. \\ \hline
Ноутбук                               & 1               & 0,05                       & 560         & 2,7                  & 74,6                           \\ \hline
Принтер                               & 1               & 0,1                        & 4           & 2,7                  & 1,08                           \\ \hline
Освещение                             & 1               & 0,6                        & 560         & 2,7                  & 907,2                          \\ \hline
\multicolumn{6}{|l|}{Всего: 983,88 руб.} \\ \hline
\end{longtable}

\subsubsection{Накладные расходы}
Расчет накладных расходов сведем в таблицу \ref{nakl_rash}.

\begin{longtable}[h!]{|*4{m{0.22\textwidth}|}}
\caption{Накладные расходы}
\label{tab:nakl_rash}
\hline
Услуга                & Количество & Стоимость одной еденицы, руб & Сумма затрат, руб. \\ \hline
Изготовление плакатов & 5 штук     & 100                          & 500                \\ \hline
Переплет              & 1 штука    & 50                           & 50                 \\ \hline
Транспортные расходы  & 10 поездок & 17                           & 170                \\ \hline
\multicolumn{4}{|l|}{Итого: 720}   \\ \hline
\end{longtable}

\subsubsection{Сводная смета затрат}
На основании всех произведенных расчетов составим сводную смету затрат на выполнение работы в виде таблицы \ref{tab:cmeta_zat}.

\begin{longtable}[h!]{|*2{m{0.45\textwidth}|}}
\caption{Сводная смета затрат}
\label{tab:cmeta_zat}
\hline
Наименование статей затрат                 & Всего, руб.       \\ \hline
ФОТ со страховыми взносами                 & 18442             \\ \hline
Основные и вспомогательные материалы       & 5478              \\ \hline
Амортизационные отчисления                 & 2520              \\ \hline
Затраты на электроэнергию                  & 983,88            \\ \hline
Накладные расходы                          & 720               \\ \hline
\multicolumn{2}{|l|}{Итого себестоимость работ: 36642,68 руб.} \\ \hline
\end{longtable}

\subsection{Научно-технический эффект}
Количественная оценка научно-технического уровня может быть произведена путем расчета результативности участников разработки по формуле:
\begin{equation}
K_{\text{ну}}=\sum (K_{de} \cdot d_i)
\end{equation}
где $K_\text{ну}$ – коэффициент научного или научно-технического уровня;

$K_\text{ну}$ – коэффициент научного или научно-технического уровня;

$K_\text{дуi}$ – коэффициент достигнутого уровня i-го фактора;

$d_i$ – значимость i-го фактора;

$n$ – количество факторов.

$K_\text{дуi}$ – коэффициент достигнутого уровня $\text{i}$-го фактора;

Весовые коэффициенты $d$ для каждого из факторов устанавливались экспертным путем. При этом сумма коэффициентов значимости по всем факторам равна единице. Коэффициенты достигнутого уровня факторов также установлены экспертным путем.

Рассчитанный коэффициент научно-технической результативности равен 0,7075. Полученное значение достаточно высоко, что говорит об эффективности проведенных работ выше среднего, однако отмечается необходимость дальнейшего развития проекта для достижения завершенности полученных результатов.

\subsection{Социальный эффект}